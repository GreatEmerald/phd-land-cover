% This is a LaTeX template for the PE&RC PhD proposal document. It has been recreated to match the DOCX file as much as possible.
% Check the comments for a few things to keep in mind.
% This document was compiled with XeLaTeX, but pdfLaTeX should work too.

\documentclass[10pt]{article}

\usepackage{fontspec}
\setsansfont{DejaVu Sans} % The template uses Verdana: DejaVu Sans, Cantarell, Open Sans, Droid Sans are all similar, so select your preferred
\renewcommand{\familydefault}{\sfdefault}

\usepackage[a4paper, left=1cm, right=1cm, top=1.3cm, bottom=1.17cm]{geometry} % Margins

\usepackage{tabularx} % Tables
\usepackage{array,booktabs} % Table borders
\usepackage{mdframed} % Table spread across multiple pages
\mdfdefinestyle{table}{frametitlerule=true, frametitlefont=\normalfont, frametitlerulewidth=1.5pt, linewidth=1.5pt}

\usepackage{enumitem} % Condensed lists

\usepackage[colorlinks, allcolors=blue]{hyperref} % URLs; can also add `xetex` if you're using XeLaTeX

%\usepackage{ulem} % Optional: strikethrough

% Header in the first page
\pagenumbering{Alph}
\usepackage{fancyhdr}
\fancyhf{} % clear all header and footer fields
\renewcommand{\headrulewidth}{0pt}
\fancyhead[R]{\rmfamily{2015.01\textbf{\thepage}}} % Based on version 2015.01
\pagestyle{fancy}

%\usepackage{natbib, graphicx, caption, enumitem, pgfgantt} % Optional: useful packages

\begin{document}

\begin{center}{\large\textbf{PE\&RC PhD PROJECT PROPOSAL}}

\textit{Please read the appendix with instructions first}\end{center}

\noindent\begin{tabularx}{\textwidth}[]{!{\vrule width 1.5pt}X|X!{\vrule width 1.5pt}}
\specialrule{1.5pt}{0pt}{0pt}
\multicolumn{2}{!{\vrule width 1.5pt}l!{\vrule width 1.5pt}}{\textbf{1. GENERAL PROJECT INFORMATION}} \\
\specialrule{1.5pt}{0pt}{0pt}
Main PE\&RC affiliated Institute / University & \\
\hline
Main PE\&RC research group & \\
\hline
Other PE\&RC groups involved & \\
\hline
Project Title (English) & \\
\hline
Project duration & FROM dd/mm/yy TO dd/mm/yy\\
\hline
Where will the research be conducted (country) & \\
\hline
At which University will the thesis be defended? & \\
\hline
Funding source(s) for this project (1, 2, or 3?) & 1 (internal) / 2 (NWO) / 3 (external) (strikethrough)\\
\hline
Name of funding source: & \\
\specialrule{1.5pt}{0pt}{0pt}
\end{tabularx}

\bigskip

\noindent\begin{tabularx}{\textwidth}[]{!{\vrule width 1.5pt}X|X!{\vrule width 1.5pt}}
\specialrule{1.5pt}{0pt}{0pt}
\multicolumn{2}{!{\vrule width 1.5pt}l!{\vrule width 1.5pt}}{\textbf{2. THE PhD CANDIDATE}} \\
\specialrule{1.5pt}{0pt}{0pt}
Full name of the PhD candidate & \\
\hline
Gender & MALE / FEMALE\\
\hline
Nationality & \\
\hline
Date of birth & \\
\hline
Period of appointment & FROM dd/mm/yy TO dd-mm-yy\\
\hline
Hours per week & \\
\specialrule{1.5pt}{0pt}{0pt}
\end{tabularx}

\bigskip

\noindent\begin{tabularx}{\textwidth}[]{!{\vrule width 1.5pt}X|X|X|X|X!{\vrule width 1.5pt}}
\specialrule{1.5pt}{0pt}{0pt}
\multicolumn{5}{!{\vrule width 1.5pt}l!{\vrule width 1.5pt}}{\textbf{3. SUPERVISION}} \\
\specialrule{1.5pt}{0pt}{0pt}
\textbf{Project role} & \textbf{Name + title} & \textbf{Specialisation} & \textbf{Organisation} & \textbf{Hours/week} \\
\hline
Promotor & \begin{minipage}[t]{\linewidth}\begin{enumerate}[nosep,after=\strut] \item \item \end{enumerate}\end{minipage} &  &  & \\
\hline
Daily supervisor & \begin{minipage}[t]{\linewidth}\begin{enumerate}[nosep,after=\strut] \item \item \end{enumerate}\end{minipage} &  &  & \\
\hline
Advisor & \begin{minipage}[t]{\linewidth}\begin{enumerate}[nosep,after=\strut] \item \item \end{enumerate}\end{minipage} &  &  & \\
\hline
Technician & \begin{minipage}[t]{\linewidth}\begin{enumerate}[nosep,after=\strut] \item \item \end{enumerate}\end{minipage} &  &  & \\
\hline
Other & \begin{minipage}[t]{\linewidth}\begin{enumerate}[nosep,after=\strut] \item \item \end{enumerate}\end{minipage} &  &  & \\
\specialrule{1.5pt}{0pt}{0pt}
\end{tabularx}

\bigskip

\noindent\begin{tabularx}{\textwidth}[]{!{\vrule width 1.5pt}p{4.8cm}|X|X!{\vrule width 1.5pt}}
\specialrule{1.5pt}{0pt}{0pt}
\multicolumn{3}{!{\vrule width 1.5pt}l!{\vrule width 1.5pt}}{\textbf{4. COLLABORATION}} \\
\specialrule{1.5pt}{0pt}{0pt}
\textbf{Type of organisation} & \textbf{Name of organisation} & \textbf{Name + title of collaborator(s)} \\
\hline
University & \begin{minipage}[t]{\linewidth}\begin{enumerate}[nosep,after=\strut] \item \item \end{enumerate}\end{minipage} & \\
\hline
Research Institute & \begin{minipage}[t]{\linewidth}\begin{enumerate}[nosep,after=\strut] \item \item \end{enumerate}\end{minipage} & \\
\hline
Government agency & \begin{minipage}[t]{\linewidth}\begin{enumerate}[nosep,after=\strut] \item \item \end{enumerate}\end{minipage} &  \\
\hline
Others (e.g., FAO,WHO) & \begin{minipage}[t]{\linewidth}\begin{enumerate}[nosep,after=\strut] \item \item \end{enumerate}\end{minipage} &  \\
\specialrule{1.5pt}{0pt}{0pt}
\end{tabularx}

\bigskip

\noindent\begin{tabularx}{\textwidth}[]{!{\vrule width 1.5pt}X|p{4.5cm}!{\vrule width 1.5pt}}
\specialrule{1.5pt}{0pt}{0pt}
\multicolumn{2}{!{\vrule width 1.5pt}l!{\vrule width 1.5pt}}{\textbf{5. ETHICS}} \\
\specialrule{1.5pt}{0pt}{0pt}
Will vertebrates be used in animal experiments? & YES / NO\\
\hline
Are there other ethical issues to be considered with respect to this project? & YES / NO\\
\hline
\multicolumn{2}{!{\vrule width 1.5pt}l!{\vrule width 1.5pt}}{If YES, please elaborate:}\\[1cm]
\specialrule{1.5pt}{0pt}{0pt}
\end{tabularx}

\newpage

\begin{center}{\large\textbf{PE\&RC PhD PROJECT PROPOSAL}}

\textit{In case a peer-reviewed full project proposal is available (e.g., NWO or EU), please send that proposal along together with the reviewers comments and the acceptance letter.}\end{center}

% Using mdframed for ``tables'' that can span multiple pages. Alternatives are also possible.
% NOTE: you cannot use figures inside `mdframed`. Instead, use a minipage, like this:
% 
%\centerline{
%  \begin{minipage}[t]{0.8\linewidth}
%    \includegraphics[width=\linewidth]{figure.pdf}
%    \captionof{figure}{Caption.} % This requires the caption package
%    \label{fig-example}
%  \end{minipage}
%}

\begin{mdframed}[style=table,frametitle=\textbf{6. SUMMARY} (max. 250 words)]
\vspace{2cm} % Remove this space when not needed
\end{mdframed}

\begin{mdframed}[style=table,frametitle=\textbf{7. DETAILED DESCRIPTION OF THE RESEARCH PLAN} (max. 2500 words + 1 page literature list)]
\vspace{2cm}
\end{mdframed}

\begin{mdframed}[style=table,frametitle=\textbf{8. TIME TABLE OF THE PROJECT AND WORK PROGRAMME}]
\vspace{2cm} % The pgfgantt package is useful here
\end{mdframed}

\begin{mdframed}[style=table,frametitle=\textbf{9. SOCIETAL RELEVANCE}]
\vspace{2cm}
\end{mdframed}

\begin{mdframed}[style=table,frametitle=\textbf{10. DATA MANAGEMENT} (max. 250 words)]
\vspace{2cm}
\end{mdframed}

\newpage

\noindent\begin{tabularx}{\textwidth}[]{!{\vrule width 1.5pt}p{1cm}|X|X|X|X!{\vrule width 1.5pt}}
\specialrule{1.5pt}{0pt}{0pt}
\multicolumn{5}{!{\vrule width 1.5pt}l!{\vrule width 1.5pt}}{\textbf{11. POTENTIAL REVIEWERS}} \\
\specialrule{1.5pt}{0pt}{0pt}
\textbf{\#} & \textbf{Name + title} & \textbf{Organisation} & \textbf{Specialisation} & \textbf{Email address} \\
\hline
1. &  &  &  & \\
\hline
2. &  &  &  & \\
\hline
3. &  &  &  & \\
\hline
4. &  &  &  & \\
\hline
5. &  &  &  & \\
\specialrule{1.5pt}{0pt}{0pt}
\end{tabularx}

\bigskip

\noindent\begin{tabularx}{\textwidth}[]{!{\vrule width 1.5pt}X|X|X!{\vrule width 1.5pt}}
\specialrule{1.5pt}{0pt}{0pt}
\multicolumn{3}{!{\vrule width 1.5pt}l!{\vrule width 1.5pt}}{\textbf{12. SIGNATURES} (this form needs to be signed by the PhD candidate as well as all supervisors)} \\
\specialrule{1.5pt}{0pt}{0pt}
\textbf{PhD candidate} & \textbf{Promotor / Principal Supervisor} & \textbf{Supervisor 2} \\
\hline
Name: & Name: & Name:\\
\hline
Date: & Date: & Date:\\ % You can use \today here
\hline
 &  &  \\[2cm]
\specialrule{1.5pt}{0pt}{0pt}
\textbf{Supervisor 3} & \textbf{Supervisor 4} & \textbf{Supervisor 5} \\
\hline
Name: & Name: & Name:\\
\hline
Date: & Date: & Date:\\
\hline
 &  &  \\[2cm]
\specialrule{1.5pt}{0pt}{0pt}
\end{tabularx}

\newpage

\begin{center}{\large\underline{\textbf{APPENDIX: EXPLANATION TO THE INDIVIDUAL QUESTIONS}}}\end{center}
\begin{tabularx}{\textwidth}[]{p{1.5cm}X}
Q 4: & Please mention the collaborating organisations in the context of this project. Only mention those collaborations which will result in joint activities such as joint publications.\\ \\
Q 5: & In some projects animals (vertebrates) may be involved or biotechnological research may be involved. In that case ethical guidelines of WU might be applicable.\\ \\
Q 6: & The short summary should be written as an explanation of the title of the research project.\\ \\
Q 7: & Elaborate your project proposal here. This should contain the following elements:
\begin{itemize}[nosep]
 \item Introduction, including history and background
 \item Objectives
 \item Hypotheses
 \item Research methodology
 \item Innovative aspects
\end{itemize} \\

Q 8: & The PhD candidate should be able to finish the thesis work within 4 years. This means that the reading version of the PhD thesis has to be submitted within 4 years. Within the work programme the following issues should be dealt with:
\begin{itemize}[nosep]
 \item In what way is appropriate supervision guaranteed?
 \begin{itemize}[nosep]
  \item What is the role of each member of the supervision team?
  \item In what way is progress evaluated?
  \item If during the project period changes will occur in the project team, in what way will supervision be continued?
 \end{itemize}
 \item In what way is execution arranged? Please specify:
 \begin{itemize}[nosep]
  \item Availability of technical equipment and facilities
  \item Availability of assistance by technical personnel
  \item Risks (e.g. weather, availability and willingness of third parties, ….).
 \end{itemize}
 \item Agreements made with collaborating organisations (question 4) and/or other PE\&RC groups, as far as important for the execution of the project.
\end{itemize}\\

Q 9: & What is the societal significance of the proposed research?\\ \\

Q 10: & This section outlines the data management plan and must encompass:
\begin{itemize}[nosep]
 \item Data storage (short term and long term storage), 
 \item Data ownership (issues with respect to ownership of data produced in this project or external data used for this project)
 \item Data sharing (agreement on who will have access to and use your (un)published data)
\end{itemize}
This section may include references to a more comprehensive (i.e. 2 to 3 pages) data management plan in which elements are outlined in more detail and can also refer to a plan at the level of a research group. Note that the full data management plan does not need to be included in this proposal, and that data collection is also part of a data management plan but is specified in section 7 and 8 of this project proposal.

For more details, see \url{https://www.wur.nl/en/Expertise-Services/WDCC/Data-Management-WDCC/Planning/Institutional-requirements.htm}.\\ \\

Q 11: & \textbf{The proposal will be sent to 3 reviewers. Please provide the names and contact details of 4-5 potential reviewers who are in no way involved in this project. A balanced representation of men and women from inside and outside the main PE\&RC affiliated institute is preferred. It is allowed, and even encouraged to verify the proposed reviewer’s willingness to provide an independent review of the proposal. This speeds up the review process.}
\end{tabularx}

\vfill

\textbf{Please submit the signed PDF of the PE\&RC PhD Project Proposal by email to the secretariat of the graduate school PE\&RC (\href{mailto:office.pe@wur.nl}{office.pe@wur.nl}) no later than 6 months after the start of the PhD project.}

\vfill

\end{document}
